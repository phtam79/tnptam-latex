%Tùy chỉnh mục lục
\patchcmd{\tableofcontents}{\contentsname}{\hspace*{4.3cm}MỤC LỤC}{}{}
\usepackage{ifoddpage}
\titlecontents{section}
[4em]{\sffamily}
{\color{blue}\contentslabel{2.3em}\color{blue}}{}
{\color{blue}\dotfill\contentspage\hspace*{1.3ex}}

\titlecontents{subsection}
[8em]{\sffamily}
{\color{violet}\contentslabel{2.3em}\color{violet}}{}
{\color{violet}\dotfill\contentspage\hspace*{1.3ex}}

\makeatletter
\renewcommand*\l@part[2]{%
	\ifnum \c@tocdepth >\m@ne
	\addpenalty{-\@highpenalty}%
	\vskip 1.0em \@plus\p@
	\setlength\@tempdima{1.5em}%
	\begingroup
	\parindent \z@ \rightskip \@pnumwidth
	\parfillskip -\@pnumwidth
	\leavevmode
	\advance\leftskip\@tempdima
	\hskip -\leftskip
	\colorbox{violet}{\strut%
		\makebox[\dimexpr\textwidth-1\fboxsep-6pt\relax][l]{%
			%\fontsize{12pt}{1pt}
			\color{white}\bfseries\fontfamily{qag}\selectfont\hspace*{0.3\textwidth} PHẦN #1%
			\nobreak\hfill\nobreak\hb@xt@\@pnumwidth{\hss #2}}}\par\smallskip
	\penalty\@highpenalty
	\endgroup
	\fi}

\renewcommand*\l@chapter[2]{%
	\ifnum \c@tocdepth >\m@ne
	\addpenalty{-\@highpenalty}%
	\vskip 1.0em \@plus\p@
	\setlength\@tempdima{1.5em}%
	\begingroup
	\parindent \z@ \rightskip \@pnumwidth
	\parfillskip -\@pnumwidth
	\leavevmode
	\advance\leftskip\@tempdima
	\hskip -\leftskip
	\colorbox{violet!80}{\strut%
		\makebox[\dimexpr\textwidth-1\fboxsep-6pt\relax][l]{%
			\sffamily%\fontsize{12pt}{1pt}
			\color{white}\bfseries\selectfont CHƯƠNG #1%
			\nobreak\hfill\nobreak\hb@xt@\@pnumwidth{\hss #2}}}\par\smallskip
	\penalty\@highpenalty
	\endgroup
	\fi}

%Tùy chỉnh các tiêu đề
\titleformat{\part}[display]
{\Large\bfseries\color{red!50!blue}}{
	\centering
	\begin{tikzpicture}
	\fill[red!50!blue] (-1,-1) rectangle (1,1);
	\draw(0,1.5)node{\fontfamily{ccr}\selectfont\Large PHẦN};
	\draw(0,0)node{\color{white}\fontsize{60pt}{1pt}\selectfont\thepart};
	\end{tikzpicture}
}
{0.5em}
{\titlerule[2.5pt]\vskip3pt\titlerule\vskip4pt\centering\fontsize{60pt}{1pt}\fontfamily{qhv}\selectfont\MakeUppercase}

\titlespacing{\chapter}{-8pt}{-20pt}{1cm}
\titleformat{\chapter}[display]
{\Large\bfseries\color{red!50!blue}}{
	\begin{tikzpicture}
	\fill[rounded corners=2mm,red!50!blue,scale=1.1] (-1,-1) rectangle (1,1);
	\draw(0,1.5)node{\fontfamily{ccr}\selectfont\large CHƯƠNG};
	\draw(0,0)node{\color{white}\fontsize{60pt}{1pt}\selectfont\thechapter};
	\draw[line width=3pt] (-1.1,-1.5)--(17.25,-1.5);
	\draw[line width=0.7pt,yshift=-0.15cm] (-1.1,-1.5)--(17.25,-1.5);
	\end{tikzpicture}
}
{-2.8cm}
{\hspace*{3cm}\fontsize{30pt}{1pt}\fontfamily{qhv}\selectfont\MakeUppercase}

\titlespacing{\section}{0pt}{0pt}{1cm}
\titleformat{\section}{\fontfamily{qag}\selectfont\color{blue}\centering\scshape\Large\bfseries}{{\fontsize{10 pt}{1pt}\selectfont\color{red} BÀI}~\fontsize{40 pt}{1pt}\selectfont\circEX{\thesection}}{-1.3cm}{
	\setcounter{vd}{0}
	\setcounter{ex}{0}
	\setcounter{bt}{0}
	\setcounter{dl}{0}
	\setcounter{dn}{0}
	\setcounter{hq}{0}
	\setcounter{tc}{0}
	\setcounter{\tcbcounter}{0} \MakeUppercase
}

\newcommand{\circlednumber}[1]{%
	\begin{tikzpicture}[baseline=(A.base),scale=1.2]%
	\node[circle,draw=none,fill=violet,inner sep=1pt,outer sep=5pt] (A) {#1};
	\end{tikzpicture}%
}
\titleformat{\subsection}{\large\bfseries\fontfamily{qag}\selectfont\color{violet}}{
	\hspace*{-0.7ex}\circlednumber{\color{white}\fontfamily{qag}\large\selectfont\thesubsection}
}{0em}{\bfseries\fontfamily{qag}\selectfont\color{violet}\MakeUppercase
}

\newcommand{\diamondEX}[1]{%
	\begin{tikzpicture}[baseline=(A.base),scale=1.2]%
	\node[diamond,draw=none,fill=violet,inner sep=1pt,outer sep=2pt] (A) {#1};
	\end{tikzpicture}%
}
\titleformat{\subsubsection}{\bfseries\color{violet}}{\diamondEX{\color{white}\fontfamily{qhv}\selectfont\thesubsubsection} }{0em}{\fontfamily{qhv}\selectfont\color{violet}
}

\renewcommand\labelitemi{\raisebox{.2\baselineskip}{%
		\rule{0.5cm}{.5pt}%
	}}
%\renewcommand{\labelenumi}{\circEX{\theenumi}}
\renewcommand{\labelenumi}{\circled{\small\fontfamily{lmss}\bfseries\selectfont\theenumi}}
\newcounter{CircLabel}
\newcommand*\CircLabel{%
	\refstepcounter{CircLabel}%
	\hskip1em\llap{\alph{CircLabel})}}%
\NewEnviron{circ_enum}[1][]{%
	\hfill
	\setcounter{CircLabel}{0}
	\def\tempbegin{\begin{taskEX}[label=\CircLabel](#1)}%
		\expandafter\tempbegin\BODY
	\end{taskEX}
}
\setlistsEX{label=\CircLabel}
\AtBeginEnvironment{listEX}{\setcounter{CircLabel}{0}}
\AtBeginEnvironment{enumEX}{\setcounter{CircLabel}{0}}
%\newcommand{\dapso}[1]{\hfill{\fontsize{8pt}{1pt}\selectfont\textbf{ĐS: }#1}}

\renewenvironment{tomtat}{
	\begin{tcolorbox}[boxrule=1pt,arc=0mm,breakable,colframe=violet,colback=white]
	}{\end{tcolorbox}}

\newtcolorbox[auto counter]{dangg}[1]{
	fonttitle=\bfseries\upshape,%fontupper=\itshape,
	colframe=violet,colback=violet!5,
	sharp corners, breakable, 
	title={\color{white}\faFolderO\fontfamily{anttlc}\selectfont \ DẠNG~\thetcbcounter.\ #1}
	\addcontentsline{toc}{subsection}{{\color{violet}Dạng~\thetcbcounter. #1}}
}
\renewenvironment{dang}[1]{
	\begin{dangg}{#1}
		\textit{Phương pháp giải:~}
	}{\end{dangg}}

\newtcolorbox{luuyy}[1][]{enhanced,
	before skip=2mm,after skip=3mm,
	boxrule=0.4pt,left=5mm,right=2mm,top=1mm,bottom=1mm,
	colback=yellow!50,
	colframe=yellow!20!black,
	sharp corners,rounded corners=southeast,arc is angular,arc=3mm,
	underlay={%
		\path[fill=tcbcol@back!80!black] ([yshift=3mm]interior.south east)--++(-0.4,-0.1)--++(0.1,-0.2);
		\path[draw=tcbcol@frame,shorten <=-0.05mm,shorten >=-0.05mm] ([yshift=3mm]interior.south east)--++(-0.4,-0.1)--++(0.1,-0.2);
		\path[fill=yellow!50!black,draw=none] (interior.south west) rectangle node[white]{\Huge\bfseries !} ([xshift=4mm]interior.north west);
	},
	drop fuzzy shadow, #1}
\renewenvironment{luuy}{\begin{luuyy}}{\end{luuyy}}

\usepackage{pifont}
\renewcommand{\loigiaiEX}{{\color{violet}\ding{45}\fontfamily{anttlc}\selectfont\bfseries~LỜI GIẢI.}}
\def\qedEX{\color{violet}\ensuremath{\square}}

\theoremstyle{}
\theoremseparator{.}
\theorembodyfont{\rm}
\renewtheorem{bt}{\color{violet}\fontfamily{anttlc}\selectfont\MakeUppercase{Bài}}
\renewtheorem{vd}{\color{violet}\fontfamily{anttlc}\selectfont\MakeUppercase{Ví dụ}}

\def\beginbox{\begin{tcolorbox}[colframe=violet,colback=white,breakable,boxrule=1pt,arc=0cm]}
\def\endbox{\end{tcolorbox}}
\AtBeginEnvironment{vd}{
	\beginbox
	\renewcommand{\loigiai}[1]{
		\renewcommand{\immini}[2]{
			\setbox\imbox=\vbox{\hbox{##2}}
			\widthimmini=\wd\imbox
			\IMleftright{##1}{##2}
		}
		\endbox
		\begin{onlysolution}
			#1
		\end{onlysolution}
		\def\endbox{}
	}
}
\AtEndEnvironment{vd}{
	\endbox
}

\theorembodyfont{\it}
\renewtheorem{dl}{ Định lí}
\theoremseparator{.}
\theorembodyfont{\rm}
\renewtheorem{hq}{ Hệ quả}
\renewtheorem{tc}{ Tính chất}
\renewtheorem{dn}{ Định nghĩa}
\theoremstyle{nonumberplain}
\renewtheorem{kn}{ Khái niệm}
\theoremseparator{}
\theorembodyfont{\it}
%Đánh số trang
\assignpagestyle{\part}{fancy}
\assignpagestyle{\chapter}{fancy}
\pagestyle{fancy}
\fancyhf{}
\renewcommand{\footrule}{{\color{violet}\vskip-\footruleskip\vskip-\footrulewidth \hrule width\headwidth height\footrulewidth\vskip\footruleskip}}
\renewcommand{\footrulewidth}{1pt}
\renewcommand{\headrulewidth}{0pt}
\renewcommand{\sectionmark}[1]{\markright{\arabic{section}. #1}}
\fancyhead[LO,RE,LE,RO]{}
\fancyfoot[RE,LO]{\color{violet}\fontsize{10 pt}{1pt}\selectfont\bfseries\leftmark}
\fancyfoot[LE]{\color{violet}\bfseries \thepage~|~Nhóm Toán và \LaTeX}
\fancyfoot[RO]{\color{violet}\bfseries\rightmark~|~\thepage}